\documentclass{ieeeaccess}
\usepackage{cite}
\usepackage{amsmath,amssymb,amsfonts}
\usepackage{algorithmic}
\usepackage{graphicx}
\usepackage{textcomp}


\usepackage{hyperref}
\usepackage{mathtools}

\usepackage{lipsum}
\usepackage{bm}
\makeatletter
\AtBeginDocument{\DeclareMathVersion{bold}
\SetSymbolFont{operators}{bold}{T1}{times}{b}{n}
\SetSymbolFont{NewLetters}{bold}{T1}{times}{b}{it}
\SetMathAlphabet{\mathrm}{bold}{T1}{times}{b}{n}
\SetMathAlphabet{\mathit}{bold}{T1}{times}{b}{it}
\SetMathAlphabet{\mathbf}{bold}{T1}{times}{b}{n}
\SetMathAlphabet{\mathtt}{bold}{OT1}{pcr}{b}{n}
\SetSymbolFont{symbols}{bold}{OMS}{cmsy}{b}{n}
\renewcommand\boldmath{\@nomath\boldmath\mathversion{bold}}}
\makeatother

\def\BibTeX{{\rm B\kern-.05em{\sc i\kern-.025em b}\kern-.08em
    T\kern-.1667em\lower.7ex\hbox{E}\kern-.125emX}}

\newcommand{\Tau}{\scalebox{1.60}{$\tau$}}
% For generic tf's
% \newcommand{\tf}[2]{\prescript{#1}{}{T_{#2}}}
\newcommand{\tf}[2]{\prescript{#1}{}{\mathbf{T}^{#2}}}
\newcommand{\tfC}[3]{\prescript{#1}{}{\mathbf{T}^{#2}_{#3}}}
\newcommand{\tfEstimated}[2]{\prescript{#1}{}{\mathbf{\hat{T}}^{#2}}}
\newcommand{\tfEstimatedC}[3]{\prescript{#1}{}{\mathbf{\hat{T}}^{#2}_{#3}}}

% For atomic tf's
\newcommand{\tfAtomic}[2]{\prescript{#1}{}{\mathbf{\Tau}^{#2}}}
\newcommand{\tfAtomicC}[3]{\prescript{#1}{}{\mathbf{\Tau}^{#2}_{#3}}}

% For atomic estimated tf's
\newcommand{\tfAtomicEstimated}[2]{\prescript{#1}{}{\mathbf{\hat{\Tau}}^{#2}}}



\newcommand{\optParam}[2]{\prescript{}{#1}{\mathbf{\hat{\Tau}}_{#2}}}

%
\newcommand{\intrisics}[1]{\mathbf{K}_{#1}}

% Maybe unnecessary but makes sure I dont screw up
\newcommand{\coordsthreeD}[2]{\mathbf{#1}_{#2}}
\newcommand{\coordstwoD}[2]{{#1}_{#2}}

\newcommand{\rotationComponent}[3]{\prescript{#1}{}{\mathbf{R}^{#2}_{#3}}}
\newcommand{\translationComponent}[3]{\prescript{#1}{}{\mathbf{t}^{#2}_{#3}}}

%Your document starts from here ___________________________________________________
\begin{document}
% \history{Date of publication xxxx 00, 0000, date of current version xxxx 00, 0000.}
% \doi{10.1109/ACCESS.2024.0429000}

\title{Annotation of Calibration Patterns for RGB-LiDAR Evaluations using ML. (Replace with name
of model)}
\author{\uppercase{Bruno Silva}\authorrefmark{1},PhD. Student\\
\uppercase{Gonçalo Ribeiro}\authorrefmark{1},PhD. Student}

\address[1]{Department of Mechanical Engineering, University of Aveiro}
% \address[2]{Department of Physics, Colorado State University, Fort Collins,
% CO 80523 USA (e-mail: author@lamar.colostate.edu)}
% \address[3]{Electrical Engineering Department, University of Colorado, Boulder, CO
% 80309 USA}

% \markboth
% {Author \headeretal: Preparation of Papers for IEEE TRANSACTIONS and JOURNALS}
% {Author \headeretal: Preparation of Papers for IEEE TRANSACTIONS and JOURNALS}


\begin{abstract}
  \lipsum[1-2]
\end{abstract}

\begin{keywords}
Extrinsic Calibration, Robot Calibration, Calibration Pattern, Machine Learning
\end{keywords}

\titlepgskip=-21pt

\maketitle

\section{Introduction}
\label{sec:introduction}

Extrinsic calibration is a fundamental process in robotics vision that involves determining the
relative pose (position and orientation) between different sensors, known as \textit{sensor to sensor calibration}, or between a sensor and a known reference
frame, which is known as \textit{sensor to coordinate frame}. This process is crucial because it allows for the accurate integration of data from multiple sensors,
enabling sensor fusion. For instance, in an autonomous vehicle, the visual information of the camera
needs to be accurately aligned with the distance measurements of the LiDAR to build a coherent understanding of the
surroundings. Similarly, in robot arms, the position of the camera position relative to the end-effector must be precisely
known to perform tasks like object manipulation. 

Usually, iterative approaches are used. These rely on a cost function specific to a sensor modality but usually suffer from ambiguities
in some way or another. An typical cost function for an RGB camera relies on computing the difference between the projection of the
detection in the 2D image of some key points into a coordinate frame where these key points are precisely known. The objects that
contain these precisely known points are called \textit{Calibration patterns}. The most common types are chessboards and \textit{ChArUcos}.
\textit{ChArUcos} are chessboards with little unique identifiable symbols that allow computer vision algorithms to decipher if the
pattern is upside down or if the framing of the image cuts them off. The issue with the aforementioned RGB cost function is that they
have multiple local minima and not always converge. A simple example is to picture a RGB camera pixed on the end of a prismatic joint
with the calibration pattern in front of the sensor, perfectly perpendicular to it. Both moving the offset of the joint or
moving the RGB sensor on the mount can lead to the same relative distance between the sensor and the pattern, thus leading to
ambiguity. 

Despite the shortcomings mentioned, these cost functions have the advantage of generally performing more accurately in larger systems,
as there is much more variety of data, and can still be used effectively in most simpler systems, as their requirements are only the
existence of a sensor and a pattern. However, this creates the need for true evaluation procedures to assess the quality of the
calibration results and to make them comparable in order to be publishable. 

This project tackles a improvement to a previously fully manual and cumbersome evaluation method between RGB and LiDAR sensors,
integrated on ATOM, a well established multi-sensor multi-modal calibration framework. The working principle is that by knowing the
physical outer limits of the pattern in the 2D image, these points can be projected into the coordinate frame of LiDAR sensor, provided
that the camera intrinsics are known. Afterward, the 3D points resulting from the projection can be directly compared with the 3D
points of the outer border of the pattern. This comparison is not ambiguous as the projected points only line up in the 3D frame if the
geometric transformations required for the projection are indeed correct. In ATOM, the 3D border points are already priorly labeled as
they are a requirement for the cost function that optimizes the pose of these sensors. However, the 2D points of the border are not
required by the RGB cost function. The current solution is a manual corner labeling method, as an automatic method struggles
with detecting orientation and deal with edge cases. This project aims to develop a automatic method to simplify the evaluation
pipeline to the user.


\section{Proposed Approach}

  \lipsum[1-2]


\section{Results}

  \lipsum[1-2]

\section{Conclusion}

  \lipsum[1-2]

\EOD
\end{document}
