\documentclass{ieeeaccess}
\usepackage{cite}
\usepackage{amsmath,amssymb,amsfonts}
\usepackage{algorithmic}
\usepackage{graphicx}
\usepackage{textcomp}

\usepackage{bm}
\makeatletter
\AtBeginDocument{\DeclareMathVersion{bold}
\SetSymbolFont{operators}{bold}{T1}{times}{b}{n}
\SetSymbolFont{NewLetters}{bold}{T1}{times}{b}{it}
\SetMathAlphabet{\mathrm}{bold}{T1}{times}{b}{n}
\SetMathAlphabet{\mathit}{bold}{T1}{times}{b}{it}
\SetMathAlphabet{\mathbf}{bold}{T1}{times}{b}{n}
\SetMathAlphabet{\mathtt}{bold}{OT1}{pcr}{b}{n}
\SetSymbolFont{symbols}{bold}{OMS}{cmsy}{b}{n}
\renewcommand\boldmath{\@nomath\boldmath\mathversion{bold}}}
\makeatother

\def\BibTeX{{\rm B\kern-.05em{\sc i\kern-.025em b}\kern-.08em
    T\kern-.1667em\lower.7ex\hbox{E}\kern-.125emX}}

%Your document starts from here ___________________________________________________
\begin{document}
% \history{Date of publication xxxx 00, 0000, date of current version xxxx 00, 0000.}
% \doi{10.1109/ACCESS.2024.0429000}

\title{Calibration of Odometry Systems in Robotic Vehicles}
\author{\uppercase{Bruno Silva}\authorrefmark{1},Msc. Student}

\address[1]{Department of Electronics, Telecommunications, and Informatics, University of Aveiro}
% \address[2]{Department of Physics, Colorado State University, Fort Collins,
% CO 80523 USA (e-mail: author@lamar.colostate.edu)}
% \address[3]{Electrical Engineering Department, University of Colorado, Boulder, CO
% 80309 USA}

% \markboth
% {Author \headeretal: Preparation of Papers for IEEE TRANSACTIONS and JOURNALS}
% {Author \headeretal: Preparation of Papers for IEEE TRANSACTIONS and JOURNALS}


\begin{abstract}
Accurate odometry is essential for autonomous navigation in robotic vehicles. Traditional encoder odometry and
visual odometry are commonly used methods, each with distinct advantages and limitations. Encoder odometry,
relying on wheel rotations, often suffers from cumulative errors and slippage. Visual odometry, which uses
camera images to estimate movement, can be affected by environmental factors such as lighting and texture.
This dissertation aims to fill a gap in the current state of the art by developing a novel methodology to calibrate
robotic systems with erroneous odometry data. Building on the \textit{Atomic Transformations Optimization Method
(ATOM)} developed by the \textit{Laboratório de Automação e Robótica} at the \textit{University of Aveiro}, this work proposes
enhancements to accommodate and correct odometry inaccuracies, by estimating the transformations provided by
these systems.
\textit{ATOM} approaches the calibration problem as an extended optimization task, estimating the poses of both sensors
and calibration patterns through a combination of indivisible geometric transformations, referred to as atomic
transformations. Unlike pairwise calibration methods, ATOM employs a sensor-to-pattern paradigm, which
significantly reduces the need for numerous error functions for each sensor pair, thereby generalizing the
calibration process and making it applicable to a wide variety of robotic systems.
The methodology is validated through extensive experiments on both a simulated robot (\textit{SOFTBOT}) and a real
robot (\textit{ZAU}). The simulation results demonstrated significant improvements in calibration accuracy, confirming the efficacy of the proposed approach under controlled conditions. However, real-world
experiments with \textit{ZAU} revealed challenges due to unexpectedly large odometry errors, which lead to the
incapability of calibrating the system. Despite these challenges, the findings contribute to advancing the field of robotic
vehicles odometry
calibration, providing a reliable approach for enhancing the performance of autonomous robotic systems.
\end{abstract}

\begin{keywords}
Enter key words or phrases in alphabetical
order, separated by commas. Autocorrelation, beamforming, communications technology, dictionary learning, feedback, fMRI, mmWave, multipath, system design, multipath, slight fault, underlubrication fault.
\end{keywords}

\titlepgskip=-21pt

\maketitle

\section{Introduction}
\label{sec:introduction}
\PARstart{T}{his} document is a template for \LaTeX. If you are reading a paper or PDF version of this document, please download the LaTeX template or the MS Word
template of your preferred publication from the IEEE Website at \underline
{https://template-selector.ieee.org/secure/templateSelec}\break\underline{tor/publicationType} so you can use it to prepare your manuscript. 
If you would prefer to use LaTeX, download IEEE's LaTeX style and sample files
from the same Web page. You can also explore using the Overleaf editor at
\underline
{https://www.overleaf.com/blog/278-how-to-use-overleaf-}\break\underline{with-ieee-collabratec-your-quick-guide-to-getting-started}\break\underline{\#.xsVp6tpPkrKM9}

IEEE will do the final formatting of your paper. If your paper is intended
for a conference, please observe the conference page limits.

\EOD
\end{document}
