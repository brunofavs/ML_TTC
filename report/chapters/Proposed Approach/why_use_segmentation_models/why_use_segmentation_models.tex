\subsection{Why use segmentation models}

% Using a more simple CNN/FC combo to find just the coordinates of the corners of the pattern was not feasible as the size of the output is not
% constant and a neural network with dynamic outputs is much more complex. More often than not, at least one corner of the pattern is
% clipped in the image. If one corner is clipped, 5 points are instead needed to draw lines on the 4 sides of the pattern. If 2 adjacent corners
% are clipped, only 3 sides are visible. The solution we found to answer this problem is to find the segmentation mask of the pattern
% instead and compute the edges afterward with classical computer vision. 

Using a simpler CNN/FC combination to directly predict the coordinates of the outer corners of the
ChArUco board was not feasible because the number of output points is not constant. Designing a
neural network with dynamic outputs would introduce significant complexity.  

In many cases, at least one of the board’s outer corners is clipped in the image. When this
happens, five points are needed instead of four to properly define the pattern’s edges. If two
adjacent corners are clipped, only three sides remain visible.  

To address this challenge, we opted to predict the segmentation mask of the pattern instead. This
allowed us to later extract the edges using classical computer vision techniques.
