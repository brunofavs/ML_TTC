\documentclass[conference]{IEEEtran}
\IEEEoverridecommandlockouts
% The preceding line is only needed to identify funding in the first footnote. If that is unneeded, please comment it out.
% \usepackage{cite}
\usepackage{amsmath,amssymb,amsfonts}
\usepackage{algorithmic}
\usepackage{graphicx}
\usepackage{textcomp}
\usepackage{xcolor}


% % Packages temporarias

% % Packages adicionadas
\usepackage[hidelinks]{hyperref}
% \usepackage{booktabs}
% % \usepackage{tikz, pgfplots}
% % \usetikzlibrary{positioning,fit,calc,shapes.geometric}
\usepackage[bibstyle=ieee,citestyle=numeric-comp]{biblatex}
\addbibresource{chapters/bibliography/references.bib}
\def\BibTeX{{\rm B\kern-.05em{\sc i\kern-.025em b}\kern-.08em
    T\kern-.1667em\lower.7ex\hbox{E}\kern-.125emX}}

\DeclareSourcemap{
  \maps[datatype=bibtex]{
    \map{
      \step[fieldsource=doi,final]
      \step[fieldset=url,null]
      \step[fieldset=urldate,null]
      \step[fieldset=isbn,null]
      \step[fieldset=issn,null]
      \step[fieldset=note,null]
    }  
  }
}
\renewcommand*{\bibfont}{\footnotesize}

% \usepackage{pgfplots, pgfkeys}
% \pgfplotsset{compat=newest}

% \pgfkeys{
%     /variables/.is family, /variables,
%     xmin/.estore in = \xmin,
%     xmax/.estore in = \xmax,
%     ymin/.estore in = \ymin,
%     ymax/.estore in = \ymax,
%     ylabel/.estore in = \ylabel,
%     title/.estore in = \toptitle,
% }

% % Command to plot training and validation curves from CSV
% \newcommand{\inserttrainvalplot}[2]{
%     \pgfkeys{/variables, #2}
%     \begin{tikzpicture}
%         \begin{axis}[
%             width=0.85\linewidth,
%             height=5.5cm,
%             xmin=\xmin, xmax=\xmax,
%             ymin=\ymin, ymax=\ymax,
%             xlabel={Epochs},
%             ylabel={\ylabel{}},
%             title={\toptitle},
%             legend pos=north east,
%             grid=major
%         ]
%             \addplot[blue, thick] table[x=Epochs, y=Train losses, col sep=comma] {#1};
%             \addlegendentry{Train}
%             \addplot[red, thick] table[x=Epochs, y=Validation losses, col sep=comma] {#1};
%             \addlegendentry{Validation}
%         \end{axis}
%     \end{tikzpicture}
% }
\usepackage{pgfplots, pgfkeys}
\pgfplotsset{compat=newest}

\pgfkeys{
    /variables/.is family, /variables,
    xmin/.estore in = \xmin,
    xmax/.estore in = \xmax,
    ymin/.estore in = \ymin,
    ymax/.estore in = \ymax,
    ylabel/.estore in = \ylabel,
    title/.estore in = \toptitle,
}

% Command to plot training and validation curves from CSV
\newcommand{\inserttrainvalplot}[4]{
    \pgfkeys{/variables, #2}
    \begin{tikzpicture}
        \begin{axis}[
            width=0.85\linewidth,
            height=5.5cm,
            xmin=\xmin, xmax=\xmax,
            ymin=\ymin, ymax=\ymax,
            xlabel={Epochs},
            ylabel={\ylabel{}},
            title={\toptitle},
            legend pos=north east,
            grid=major
        ]
            \addplot[blue, thick] table[x=Epochs, y={#3}, col sep=comma] {#1};
            \addlegendentry{Train}
            \addplot[red, thick] table[x=Epochs, y={#4}, col sep=comma] {#1};
            \addlegendentry{Validation}
        \end{axis}
    \end{tikzpicture}
}

\begin{document}

% \title{Single and multi-tasked neural networks for driving assistance applications: selection and deployment\\
\title{Annotation of Calibration Patterns for RGB-LiDAR Evaluations using Segmentation Models\\
{}
}

\author{\IEEEauthorblockN{Gonçalo Ribeiro}
\IEEEauthorblockA{\textit{Department of Mechanical Engineering} \\
\textit{University of Aveiro}\\
Aveiro, Portugal \\
% gribeiro@ua.pt
}
\and
\IEEEauthorblockN{Gonçalo Ribeiro}
\IEEEauthorblockA{\textit{Department of Mechanical Engineering} \\
\textit{University of Aveiro}\\
Aveiro, Portugal \\
% gribeiro@ua.pt
}
}

% \author{\IEEEauthorblockN{}
% \IEEEauthorblockA{\textit{} \\
% \textit{}\\
% \\
% \textit{}}
% \and
% \IEEEauthorblockN{}
% \IEEEauthorblockA{\textit{} \\
% \textit{}\\
% \\
% \textit{}}
% }

\maketitle

\begin{abstract}
    Extrinsic calibration is a crucial step in sensor fusion. This work presents a deep-learning-based augmentation to ATOM, a
    multi-sensor and multi-modal calibration framework, for automating the annotation of the outer borders of calibration patterns,
    eliminating the need for manual labeling. Instead of predicting the outer corners directly with a CNN/FC combo, which proved
    infeasible due to variable output sizes, we trained segmentation models to detect the pattern and later extracted the edges using
  classical computer vision techniques. After testing various architectures, a U-Net model with a pretrained ResNet50 backbone delivered
  the best results.
\end{abstract}

\begin{IEEEkeywords}
Extrinsic Calibration, Robot Calibration, Calibration Pattern, Machine Learning
\end{IEEEkeywords}


\section{Introduction}
\label{sec:introduction}

Extrinsic calibration is a fundamental process in robotics vision that involves determining the
relative pose (position and orientation) between different sensors, known as \textit{sensor to sensor calibration}, or between a sensor and a known reference
frame, which is known as \textit{sensor to coordinate frame}. Extrinsic calibration is crucial because it allows for the accurate integration of data from multiple sensors,
enabling sensor fusion. For instance, in an autonomous vehicle, the visual information of the camera
needs to be accurately aligned with the distance measurements of the LiDAR to build a coherent understanding of the
surroundings. Similarly, in robot arms, the position of the camera position relative to the end-effector must be precisely
known to perform tasks like object manipulation. 

Usually, iterative approaches are used. These rely on a cost function specific to a sensor modality but usually suffer from ambiguities
in some way or another. An typical cost function for an RGB camera relies on computing the difference between the projection of the
detection in the 2D image of some key points into a coordinate frame where these key points are precisely known. The objects that
contain these precisely known points are called \textit{Calibration patterns}. The most common types are chessboards and \textit{ChArUcos}.
\textit{ChArUcos} are chessboards with unique identifiable symbols embedded on each square that allow computer vision algorithms to decipher if the
pattern is upside down or if the framing of the image cuts part of the calibration object off. The issue with the aforementioned RGB cost function is that they
have multiple local minima and not always converge. A simple example is to picture an RGB camera fixed on the end of a prismatic joint
with the calibration pattern in front of the sensor, perfectly perpendicular to it. Both moving the offset of the joint or
moving the RGB sensor on the mount can lead to the same relative distance between the sensor and
the pattern, leading to equal detections of the key points, thus leading to
ambiguity. 

Despite the shortcomings mentioned, these cost functions have the advantage of generally performing more accurately in larger systems,
as there is much more variety of data, and can still be used effectively in most simpler systems, as their requirements are only the
existence of a sensor and a pattern. However, this creates the need for true evaluation procedures to assess the quality of the
calibration results and to make them comparable in order to be publishable. 

This project tackles an improvement to a previously fully manual and cumbersome evaluation method between RGB and LiDAR sensors,
integrated on ATOM, a well established multi-sensor multi-modal calibration framework in the
scientific community. The working principle is that by knowing the
physical outer limits of the pattern in the 2D image, these points can be projected into the coordinate frame of LiDAR sensor, provided
that the camera intrinsics are known. Afterward, the 3D points resulting from the projection can be directly compared with the 3D
points of the outer border of the pattern. This comparison is not ambiguous as the projected points only line up in the 3D frame if the
geometric transformations required for the projection are indeed correct. In ATOM, the 3D border points are already priorly labeled as
they are a requirement for the cost function that optimizes the pose of the 3D LiDAR sensors. However, the 2D points of the border are not
required by the RGB cost function. The current solution is a manual border labeling method, as a
simple automatic method would struggle with detecting orientation and deal with edge cases. This
project aims to develop an automatic method using deep learning to simplify the evaluation pipeline to the user. \autoref{fig:input}
and \autoref{fig:output} represent an example input and a desired output. On the output, each color encodes the border of one of the
sides of the calibration pattern.

\begin{figure}[h]
    \centering
    \includegraphics[width=0.8\linewidth]{resources/images/pattern_28.jpg}
    \caption{Input image}
    \label{fig:input}
\end{figure}

\begin{figure}[h]
    \centering
    \includegraphics[width=0.8\linewidth]{resources/images/pattern_28_lines.png}
    \caption{Desired Output, where red:top ; green:right; blue:bottom, green:right.}
    \label{fig:output}
\end{figure}




% % Example usage for loss plot
% \begin{figure}[ht]
%     \centering
%     \inserttrainvalplot{resources/data/train_info_resnet_frozzen_200.csv}{ylabel=Loss, xmin=0, xmax=100, ymin=0, ymax=2, title=Training and Validation Loss}
%     \caption{Training and Validation Loss Curves}
%     \label{fig:loss_curve}
% \end{figure}

% % Example usage for DICE score plot
% \begin{figure}[ht]
%     \centering
%     \inserttrainvalplot{resources/data/train_info_resnet_frozzen_200.csv}{ylabel=DICE Score, xmin=0, xmax=100, ymin=0, ymax=1, title=Training and Validation DICE Score}
%     \caption{Training and Validation DICE Score Curves}
%     \label{fig:dice_curve}
% \end{figure}

% Example usage for loss plot
\begin{figure}[ht]
    \centering
    \inserttrainvalplot{resources/data/train_info_resnet_frozzen_200.csv}{ylabel=Loss, xmin=0, xmax=200, ymin=0, ymax=2, title=Training and Validation Loss}{Train losses}{Validation losses}
    \caption{Training and Validation Loss Curves}
    \label{fig:loss_curve}
\end{figure}
\begin{figure}[ht]
    \centering
    \inserttrainvalplot{resources/data/train_info_resnet_frozzen_200_smooth.csv}{ylabel=Loss, xmin=0, xmax=200, ymin=0, ymax=2, title=Training and Validation Loss}{Train losses}{Validation losses}
    \caption{Training and Validation Loss Curves}
    \label{fig:loss_curve_smooth}
\end{figure}

% Example usage for DICE score plot
\begin{figure}[ht]
    \centering
    \inserttrainvalplot{resources/data/train_info_resnet_frozzen_200.csv}{ylabel=DICE Score, xmin=0, xmax=200, ymin=0, ymax=1, title=Training and Validation DICE Score}{Train DICE}{Validation DICE}
    \caption{Training and Validation DICE Score Curves}
    \label{fig:dice_curve}
\end{figure}
\begin{figure}[ht]
    \centering
    \inserttrainvalplot{resources/data/train_info_resnet_frozzen_200_smooth.csv}{ylabel=DICE Score, xmin=0, xmax=200, ymin=0, ymax=1, title=Training and Validation DICE Score}{Train DICE}{Validation DICE}
    \caption{Training and Validation DICE Score Curves}
    \label{fig:dice_curve_smooth}
\end{figure}


% Example usage for DICE score plot
\begin{figure}[ht]
    \centering
    \inserttrainvalplot{resources/data/train_info_resnet_frozzen_200.csv}{ylabel=DICE Score, xmin=0, xmax=88, ymin=0, ymax=1, title=Training and Validation DICE Score}{Train DICE}{Validation DICE}
    \caption{Training and Validation DICE Score Curves}
    \label{fig:dice_curve_best}
\end{figure}
\begin{figure}[ht]
    \centering
    \inserttrainvalplot{resources/data/train_info_resnet_frozzen_200_smooth.csv}{ylabel=DICE Score, xmin=0, xmax=88, ymin=0, ymax=1, title=Training and Validation DICE Score}{Train DICE}{Validation DICE}
    \caption{Training and Validation DICE Score Curves}
    \label{fig:dice_curve_smooth_best}
\end{figure}
\section{Proposed Approach}
\subsection{Data preparation}

The data we had:
  -  Images with corners labeled

Captured more images to have more dataset diversity with other types of calibration patterns

Developed a annotation program to label corners of the patterns. 
Developed a program to convert labeled corners into segmentation masks that served later to train the models.

\subsection{Why use segmentation models}

Using a more simple CNN/FC combo to find just the coordinates of the corners of the pattern was not feasible as the size of the output is not
constant and a neural network with dynamic outputs is much more complex. More often than not, at least one corner of the pattern is
clipped in the image. If one corner is clipped, 5 points are instead needed to draw lines on the 4 sides of the pattern. If 2 adjacent corners
are clipped, only 3 sides are visible. The solution we found to answer this problem is to find the segmentation mask of the pattern
instead and compute the edges afterward with classical computer vision. 


\subsection{Model selection and training setup}
To select the models, we prioritized the availability of source code and 
pre-trained weights. The PyTorch Vision framework currently provides three 
models for semantic segmentation: DeepLabV3~\cite{DeepLabV3}, FCN~\cite{fcn}, 
and LRASPP~\cite{LRASPP}. Among these, we selected DeepLabV3, as the authors 
reported the best results.

In addition to the PyTorch models, we sought a simple and well-established model 
to serve as a baseline. For this purpose, we chose U-Net~\cite{unet}.

Since this problem involves only one class of interest—the checkerboard—with all 
other pixels considered background, it can be framed as a binary classification 
task at the pixel level. Consequently, we selected a loss function designed for 
binary classification: binary cross-entropy loss.

For optimization, we experimented with the Adam optimizer due to its fast 
convergence and adaptive learning rates. Additionally, we evaluated Adam with 
Weight Decay to mitigate potential overfitting.

For batch size, we used the maximum value that the GPU could accommodate.
\subsection{Transfer Learning with DeepLabV3 and Resnet50 Backbone}

Using transfer learning with a pretrained DeepLabv3 model was the first option considered. DeepLabV3 is a complex model designed for
image segmentation. Including the Resnet50 backbone, which is a CNN, the model has in total around 41M parameters. Upon freezing the
backbone, we were left with 17M trainable parameters. The pretrained model came from PyTorch built-in models and was originally trained
on a subset of known COCO dataset, using only the 20 categories that are present in the Pascal VOC dataset, mainly consisting of big
every objects like cars, planes, sofas, bicycle...

The train was conducted using the Adam optimizer and cross entropy loss, batch size of 4 and 10 epochs. Our dataset had 80 images. We
found the loss to stay constant and high, meaning our model was not learning anything from the training. We figured that the dataset
that the model was originally trained on was too different from ours and the model could not learn within a feasible number of epochs
with only 80 images. Our next attempt was to try a simpler segmentation model and train it from the ground up.

\subsection{U-net trained from scratch}

Training the previous model from scratch would require a lot of compute resources and probably be overkill for the desired task. As
such, we decided to try a simpler model and train it from scratch. Unet is blablabla

Good results


\subsection{U-net with pretrained Resnet50 Backbone}

Better results

\section{Results}


\begin{frame}[c]{DeepLabV3}{Com backbone Resnet50}

    \begin{columns}
      \column{0.4\textwidth}
            \begin{itemize}
                  \item 41M Total Parameters
                  \item 17M Trainable Parameters
            \end{itemize}
      \column{0.5\textwidth}
            \begin{itemize}
                  \item Model did not converge
                  \item Complications with training
            \end{itemize}
    \end{columns}

    
  \end{frame}


\begin{frame}[c]{U-net}{Treinada do raíz}

  \begin{columns}

    \column{0.5\textwidth}
      \begin{itemize}
            \item 33M de parâmetros
            \item 33M treinaveis
      \end{itemize}

    \column{0.5\textwidth}
    \begin{itemize}
      \item Exatidão de 76\% \\Época 157
    \end{itemize}


  \end{columns}

\end{frame}

\newcommand{\plotA}{\inserttrainvalplot{resources/data/train_info_unet_200_smooth.csv}{ylabel=Loss, xmin=0, xmax=157, ymin=0, ymax=2, title=U-Net from scratch - Smoothed losses}{Train losses}{Validation losses}{north east}}
\newcommand{\plotB}{\inserttrainvalplot{resources/data/train_info_unet_200_smooth.csv}{ylabel=DICE Score, xmin=0, xmax=157, ymin=0, ymax=1, title=U-Net from scratch - Smoothed DICE scores}{Train DICE}{Validation DICE}{north west}}

% \begin{frame}[c]{U-net}{Treinada do raíz}

%     \begin{figure}[ht]
%         \centering
%         \inserttrainvalplot{resources/data/train_info_unet_200_smooth.csv}{ylabel=Loss, xmin=0, xmax=157, ymin=0, ymax=2, title=U-Net from scratch - Smoothed losses}{Train losses}{Validation losses}{north east}
%     \end{figure}
% \end{frame}

% \begin{frame}[c]{U-net}{Treinada do raíz}

%     \begin{figure}[H]
%         \centering
%         \inserttrainvalplot{resources/data/train_info_unet_200_smooth.csv}{ylabel=DICE Score, xmin=0, xmax=157, ymin=0, ymax=1, title=U-Net from scratch - Smoothed DICE scores}{Train DICE}{Validation DICE}{north west}
%     \end{figure}

% \end{frame}

\begin{frame}[c]{U-net}{Treinada do raíz}

    \begin{figure}[ht]
        \centering
        \only<1>\plotA
        \only<2>\plotB
    \end{figure}

\end{frame}


\begin{frame}[c]{U-net}{Com Resnet50 Backbone}

    \begin{columns}

        \column{0.5\textwidth}
          \begin{itemize}
                \item 148M de parâmetros
                \item 124M treinaveis
          \end{itemize}
    
        \column{0.5\textwidth}
        \begin{itemize}
          \item Exatidão de 85\% \\Época 142
        \end{itemize}
    
    
      \end{columns}
    
  \end{frame}

%   \begin{frame}[c]{U-net}{Com Resnet50 Backbone}
%     \begin{figure}[ht]
%         \centering
%         \inserttrainvalplot{resources/data/train_info_resnet_frozzen_200_smooth.csv}{ylabel=Loss, xmin=0, xmax=142, ymin=0, ymax=2, title=U-Net with ResNet50 (frozen) - Smoothed losses}{Train losses}{Validation losses}{north east}
%     \end{figure}
    
    
%   \end{frame}

%   \begin{frame}[c]{U-net}{Com Resnet50 Backbone}
%     \begin{figure}[ht]
%         \centering
%         \inserttrainvalplot{resources/data/train_info_resnet_frozzen_200_smooth.csv}{ylabel=DICE Score, xmin=0, xmax=142, ymin=0, ymax=1, title=U-Net with ResNet50 (frozen) - Smoothed DICE scores}{Train DICE}{Validation DICE}{south east}
%     \end{figure}
    
    
%   \end{frame}
\newcommand{\plotC}{\inserttrainvalplot{resources/data/train_info_resnet_frozzen_200_smooth.csv}{ylabel=Loss, xmin=0, xmax=142, ymin=0, ymax=2, title=U-Net with ResNet50 (frozen) - Smoothed losses}{Train losses}{Validation losses}{north east}}
\newcommand{\plotD}{\inserttrainvalplot{resources/data/train_info_resnet_frozzen_200_smooth.csv}{ylabel=DICE Score, xmin=0, xmax=142, ymin=0, ymax=1, title=U-Net with ResNet50 (frozen) - Smoothed DICE scores}{Train DICE}{Validation DICE}{south east}}

\begin{frame}[c]{U-net}{Com Resnet50 Backbone}
\begin{figure}[ht]
    \centering
    \only<1>\plotC
    \only<2>\plotD
\end{figure}

\end{frame}


  \begin{frame}[c]{U-net}{Com Resnet50 Backbone}

    \begin{columns}

        \column{0.5\textwidth}
          \begin{itemize}
                \item 148M de parâmetros
                \item 148M treinaveis
          \end{itemize}
    
        \column{0.5\textwidth}
        \begin{itemize}
          \item Exatidão de 85\% \\Época 17
        \end{itemize}
    
    
      \end{columns}
    
  \end{frame}
\newcommand{\plotone}{\inserttrainvalplot{resources/data/train_info_resnet_unfrozzen_200_smooth.csv}{ylabel=Loss, xmin=0, xmax=17, ymin=0, ymax=2, title=U-Net with ResNet50 (unfrozen) - Smoothed losses}{Train losses}{Validation losses}{north east}}
\newcommand{\plottwo}{\inserttrainvalplot{resources/data/train_info_resnet_unfrozzen_200_smooth.csv}{ylabel=DICE Score, xmin=0, xmax=17, ymin=0, ymax=1, title=U-Net with ResNet50 (unfrozen) - Smoothed DICE scores}{Train DICE}{Validation DICE}{south east}}
  \begin{frame}[c, fragile]{U-net}{Com Resnet50 Backbone}
    \begin{figure}[ht]
        \centering
        % \only<1>\inserttrainvalplot{resources/data/train_info_resnet_unfrozzen_200_smooth.csv}{ylabel=Loss, xmin=0, xmax=17, ymin=0, ymax=2, title=U-Net with ResNet50 (unfrozen) - Smoothed losses}{Train losses}{Validation losses}{north east}
        % \only<2>\inserttrainvalplot{resources/data/train_info_resnet_unfrozzen_200_smooth.csv}{ylabel=DICE Score, xmin=0, xmax=17, ymin=0, ymax=1, title=U-Net with ResNet50 (unfrozen) - Smoothed DICE scores}{Train DICE}{Validation DICE}{south east}
        \only<1>\plotone
        \only<2>\plottwo
        
    \end{figure}

  \end{frame}

%   \begin{frame}[c]{U-net}{Com Resnet50 Backbone}
%     \begin{figure}[ht]
%         \centering
%         \inserttrainvalplot{resources/data/train_info_resnet_unfrozzen_200_smooth.csv}{ylabel=DICE Score, xmin=0, xmax=17, ymin=0, ymax=1, title=U-Net with ResNet50 (unfrozen) - Smoothed DICE scores}{Train DICE}{Validation DICE}{south east}
%     \end{figure}
    
%   \end{frame}

\begin{frame}[c]{Alguns resultados do melhor modelo}
    \begin{figure}[ht]
        \centering
        \begin{minipage}[b]{0.45\linewidth}
            \centering
            \only<1>{\includegraphics[width=1\textwidth]{resources/images/gtruth/pattern_58.jpg}}
            \only<2>{\includegraphics[width=1\textwidth]{resources/images/gtruth/pattern_61.jpg}}
            \only<3>{\includegraphics[width=1\textwidth]{resources/images/gtruth/pattern_63.jpg}}
            \only<4>{\includegraphics[width=1\textwidth]{resources/images/gtruth/rgbd_hand_color_190.jpg}}
            \captionsetup{labelformat=empty}
            \caption{Resultado esperado}
        \end{minipage}
        \hspace{0.5cm}
        \begin{minipage}[b]{0.45\linewidth}
            \centering
            \only<1>{\includegraphics[width=1\textwidth]{resources/images/preds/Small_dataset_unet_resnet/pattern_58.jpg}}
            \only<2>{\includegraphics[width=1\textwidth]{resources/images/preds/Small_dataset_unet_resnet/pattern_61.jpg}}
            \only<3>{\includegraphics[width=1\textwidth]{resources/images/preds/Small_dataset_unet_resnet/pattern_63.jpg}}
            \only<4>{\includegraphics[width=1\textwidth]{resources/images/preds/Small_dataset_unet_resnet/rgbd_hand_color_190.jpg}}
            \captionsetup{labelformat=empty}
            \caption{Resultado obtido}
        \end{minipage}
    \end{figure}
\end{frame}


\section{Conclusion}

We presented a approach to automate the annotation of the outer borders of a calibration patterns using deep learning. By leveraging
segmentation models instead of direct CNN/FC combos, we addressed challenges related to pattern occlusions and variable output sizes.
While transfer learning with DeepLabV3 did not yield satisfactory results, training a U-Net model with a ResNet50 backbone proved
highly effective. This approach streamlines the evaluation process, reducing manual effort while maintaining accuracy. Future work
could focus on further refining the model.


\input{chapters/bibliography/references.tex}

\end{document}
