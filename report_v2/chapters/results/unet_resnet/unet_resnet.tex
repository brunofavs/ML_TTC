\subsection{U-net with pretrained Resnet50 Backbone}

After experimenting with training a U-Net model from scratch, we explored modifying its backbone 
architecture by incorporating a pretrained ResNet50. Similar to our initial approach, we sought 
an existing implementation of this configuration and adapted one found on \href{https://github.com/rawmarshmellows/pytorch-unet-resnet-50-encoder}{GitHub}\footnote{\url{https://github.com/rawmarshmellows/pytorch-unet-resnet-50-encoder}}. 

However, there was an unexpected discrepancy in the parameter count. The original U-Net model from 
scratch had 31 million parameters, but adding a ResNet50 backbone—which should have contributed 
approximately 50 million parameters—resulted in a total of 148 million parameters instead of the 
expected ~80 million. This suggests that either this or the previous implementation may have 
contained an error.

Despite this issue, we began by training the model with the ResNet50 backbone's parameters frozen, 
leaving us with 124 million trainable parameters. Ideally, only the 50 million backbone parameters 
should have been frozen, but our approach was imperfect. Nevertheless, we proceeded with the training 
using a batch size of 19 and processing images at a resolution of 512x512 pixels.

The resulting training process was significantly more stable compared to the U-Net trained from 
scratch, as demonstrated in Figures \ref{fig:renet50_frozen_loss_curve} and \ref{fig:renet50_frozen_dice_curve}.

% Frozen backbone
% 124300609

\begin{figure}[ht]
    \centering
    \inserttrainvalplot{resources/data/train_info_resnet_frozzen_200.csv}{ylabel=Loss, xmin=0, xmax=200, ymin=0, ymax=2, title=Training and Validation Loss}{Train losses}{Validation losses}{north east}
    \caption{Training and Validation Loss Curves}
    \label{fig:renet50_frozen_loss_curve}
\end{figure}

\begin{figure}[ht]
    \centering
    \inserttrainvalplot{resources/data/train_info_resnet_frozzen_200.csv}{ylabel=DICE Score, xmin=0, xmax=200, ymin=0, ymax=1, title=Training and Validation DICE Score}{Train DICE}{Validation DICE}{south east}
    \caption{Training and Validation DICE Score Curves}
    \label{fig:renet50_frozen_dice_curve}
\end{figure}

The model achieved a peak validation DICE score of 85\% at epoch 142. Similarly to the U-Net from scratch, 
for enhanced clarity in visualizing the training dynamics, Figures \ref{fig:renet50_frozen_loss_curve_smooth} and 
\ref{fig:renet50_frozen_dice_curve_smooth} present smoothed versions of the loss and DICE curves up to epoch 142.

\begin{figure}[ht]
    \centering
    \inserttrainvalplot{resources/data/train_info_resnet_frozzen_200_smooth.csv}{ylabel=Loss, xmin=0, xmax=142, ymin=0, ymax=2, title=Training and Validation Loss}{Train losses}{Validation losses}{north east}
    \caption{Training and Validation Loss Curves}
    \label{fig:renet50_frozen_loss_curve_smooth}
\end{figure}

\begin{figure}[ht]
    \centering
    \inserttrainvalplot{resources/data/train_info_resnet_frozzen_200_smooth.csv}{ylabel=DICE Score, xmin=0, xmax=142, ymin=0, ymax=1, title=Training and Validation DICE Score}{Train DICE}{Validation DICE}{south east}
    \caption{Training and Validation DICE Score Curves}
    \label{fig:renet50_frozen_dice_curve_smooth}
\end{figure}

In the final experiment, we trained all 148 million parameters without freezing, using a batch size of 14 and 512x512 pixel images. The outcomes are shown in Figures \ref{fig:renet50_unfrozen_loss_curve} and \ref{fig:renet50_unfrozen_dice_curve}.

% Unfrozen backbone
% 147808641

\begin{figure}[ht]
    \centering
    \inserttrainvalplot{resources/data/train_info_resnet_unfrozzen_200.csv}{ylabel=Loss, xmin=0, xmax=200, ymin=0, ymax=3, title=Training and Validation Loss}{Train losses}{Validation losses}{north east}
    \caption{Training and Validation Loss Curves}
    \label{fig:renet50_unfrozen_loss_curve}
\end{figure}

\begin{figure}[ht]
    \centering
    \inserttrainvalplot{resources/data/train_info_resnet_unfrozzen_200.csv}{ylabel=DICE Score, xmin=0, xmax=200, ymin=0, ymax=1, title=Training and Validation DICE Score}{Train DICE}{Validation DICE}{south east}
    \caption{Training and Validation DICE Score Curves}
    \label{fig:renet50_unfrozen_dice_curve}
\end{figure}

The model achieved a peak validation DICE score of 85\% at epoch 17. Surprisingly, this peak was reached much earlier 
than in the other experiments. When examining the smoothed results up to epoch 17, as shown in Figures 
\ref{fig:renet50_unfrozen_loss_curve_smooth} and \ref{fig:renet50_unfrozen_dice_curve_smooth}, the 
training appears significantly more stable.

\begin{figure}[ht]
    \centering
    \inserttrainvalplot{resources/data/train_info_resnet_unfrozzen_200_smooth.csv}{ylabel=Loss, xmin=0, xmax=17, ymin=0, ymax=2, title=Training and Validation Loss}{Train losses}{Validation losses}{north east}
    \caption{Training and Validation Loss Curves}
    \label{fig:renet50_unfrozen_loss_curve_smooth}
\end{figure}


\begin{figure}[ht]
    \centering
    \inserttrainvalplot{resources/data/train_info_resnet_unfrozzen_200_smooth.csv}{ylabel=DICE Score, xmin=0, xmax=17, ymin=0, ymax=1, title=Training and Validation DICE Score}{Train DICE}{Validation DICE}{south east}
    \caption{Training and Validation DICE Score Curves}
    \label{fig:renet50_unfrozen_dice_curve_smooth}
\end{figure}
