\subsection{U-net trained from scratch}
After encountering difficulties with DeepLabV3, we shifted our approach to training a U-Net model from scratch. 
Given the complexity of implementing U-Net directly from its original paper with our current expertise, we opted 
instead to find an existing implementation. We adapted one from an \href{https://medium.com/@fernandopalominocobo/mastering-u-net-a-step-by-step-guide-to-segmentation-from-scratch-with-pytorch-6a17c5916114}{online tutorial}\footnote{\href{https://medium.com/@fernandopalominocobo/mastering-u-net-a-step-by-step-guide-to-segmentation-from-scratch-with-pytorch-6a17c5916114}{https://medium.com/@fernandopalominocobo/mastering-u-net-a-step-by-step-guide-to-segmentation-from-scratch-with-pytorch-6a17c5916114}}.
to suit our needs. 

The chosen U-Net model comprised 31 million trainable parameters, utilizing a batch size of 20 and processing 
images at a resolution of 512x512 pixels. While the training process was notably unstable, as evidenced by Figures 
\ref{fig:unet_loss_curve} and \ref{fig:unet_dice_curve}, which illustrate the loss and DICE curves, it was far 
better than DeepLabV3. The model achieved a peak validation DICE score of 78\% at epoch 157. 

\begin{figure}[ht]
    \centering
    \inserttrainvalplot{resources/data/train_info_unet_200.csv}{ylabel=Loss, xmin=0, xmax=200, ymin=0, ymax=2, title=U-Net from scratch - Losses}{Train losses}{Validation losses}{north east}
    \caption{Training and validation loss curves of the U-Net model trained from scratch for 200 epochs.}
    \label{fig:unet_loss_curve}
\end{figure}

\begin{figure}[ht]
    \centering
    \inserttrainvalplot{resources/data/train_info_unet_200.csv}{ylabel=DICE Score, xmin=0, xmax=200, ymin=0, ymax=1, title=U-Net from scratch - DICE scores}{Train DICE}{Validation DICE}{north west}
    \caption{Training and validation DICE score curves of the U-Net model trained from scratch for 200 epochs.}
    \label{fig:unet_dice_curve}
\end{figure}


For enhanced clarity in visualizing the training dynamics, Figures \ref{fig:unet_loss_curve_smooth} and 
\ref{fig:unet_dice_curve_smooth} present smoothed versions of the loss and DICE curves up to epoch 157. 
The smoothing was achieved using a Savitzky-Golay filter, which helps in highlighting trends over noise. 


\begin{figure}[ht]
    \centering
    \inserttrainvalplot{resources/data/train_info_unet_200_smooth.csv}{ylabel=Loss, xmin=0, xmax=157, ymin=0, ymax=2, title=U-Net from scratch - Smoothed losses}{Train losses}{Validation losses}{north east}
    \caption{Training and validation loss curves of the U-Net model trained from scratch until reaching the maximum DICE score (157 epochs).}
    \label{fig:unet_loss_curve_smooth}
\end{figure}

\begin{figure}[H]
    \centering
    \inserttrainvalplot{resources/data/train_info_unet_200_smooth.csv}{ylabel=DICE Score, xmin=0, xmax=157, ymin=0, ymax=1, title=U-Net from scratch - Smoothed DICE scores}{Train DICE}{Validation DICE}{north west}
    \caption{Training and validation DICE score curves of the U-Net model trained from scratch until reaching the maximum DICE score (157 epochs).}
    \label{fig:unet_dice_curve_smooth}
\end{figure}